\documentclass[11pt,a4paper,twoside,titlepage]{book}
%\documentclass[11pt,a4paper,twoside,titlepage,draft]{book}

\usepackage[utf8]{inputenc}
\usepackage[OT2, T1]{fontenc}
\usepackage{lmodern}

% das Paket lmodern aktiviert den Schriftsatz Latin Modern
% hier ein paar Links, die ich vor langer Zeit zu dem Thema notiert habe
% https://tex.stackexchange.com/questions/1390/latin-modern-vs-cm-super
% https://tug.org/FontCatalogue/latinmodernroman/
% https://www.ctan.org/tex-archive/fonts/lm/
 
\usepackage[english,main=german]{babel}


\begin{document}

\frontmatter
\chapter{Abstract}%\label{abstract}
Diese LaTeX-Vorlage soll einen einfachen Ursprung haben, in der Form eines Hallo-Welt-Dokuments. Ich möchte die Vorlage daraufhin Schritt für Schritt wachsen und komplexer werden lassen. Da ich das ganze as Git-Repositorium ins Netz stelle, kann jede Leserin und jeder Leser die Wachstumsschritte Commit für Commit nachvollziehen. Ich will in den Commitnotizen hilfreiche kurze Beschreibungen dessen geben, was in jedem Beitrag neu hinzu kommt.

\mainmatter
\chapter{Zunächst mal ein Bisschen Text}%\label{nur_text}
Halli Hallo.


\end{document}
 
