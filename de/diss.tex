\documentclass[11pt,a4paper,twoside,titlepage]{book}
%\documentclass[11pt,a4paper,twoside,titlepage,draft]{book}


%-------  Mathe -- ein paar übliche Verdächtige
% AMS steht für American Mathematical Society, von dort bekommt man umfassende Sammlungen von Symbolen
% AMS-Pakete müssen vor fontec geladen werden laut lshort.pdf v5.01, Fußnote auf Seite 33
\usepackage{amsmath}		% grundlegender Mathekram, z.B. bündig angeordnete Gleichungssysteme übere mehrere Zeilen
\usepackage{amsfonts}		% Schriftsätze der American Mathematical Society
\usepackage{amssymb}		% Symbole der American Mathematical Society
\usepackage{dsfont}			% double-stroke z.B. bekommt man mit \mathds{R} ein ansehnliches Symbol für reelle Zahlen
%\usepackage{wasysym}		% geschlossene Integrale per \oint, \oiint (allerdings sehr aufrecht und nicht so schön wie AMS-Integrale)
%\usepackage{esint}			% geschlossene Integrale per \oint, \oiint, \varoiint, sehen besser aus als von wasysym

% ich konnte beobachten, dass schon bevor ich das Paket amsmath je per \usepackage{amsmath}
% angefordert habe, es schon vom Kompilierer geladen wurde

% Kontextnotiz: ich kompiliere unter Linux (Kubuntu 18.04),
% derzeit einfach noch in der Bashkonsole mit dem Befehl: pdflatex diss.tex

%-------  Codierung des Quellcodes
% ich arbeite unter Linux, vielleicht muss hier für Windows etwas angepasst werden
\usepackage[utf8]{inputenc}
\usepackage[OT2, T1]{fontenc}

%-------  Auswahl des Schriftsatzes
% das Paket lmodern aktiviert den Schriftsatz Latin Modern
% hier ein paar Links, die ich vor langer Zeit zu dem Thema notiert habe
% https://tex.stackexchange.com/questions/1390/latin-modern-vs-cm-super
% https://tug.org/FontCatalogue/latinmodernroman/
% https://www.ctan.org/tex-archive/fonts/lm/
\usepackage{lmodern}
 
%-------  document languages
\usepackage[english,main=german]{babel}
\usepackage{lgreek}			% griechisch schreiben mit \begin{greek} und \end{greek}

%-------  Einheiten, chemische Formeln, Isotope
\usepackage[decimalsymbol=comma]{siunitx}
\usepackage{isotope}
\usepackage[version=4]{mhchem} % erleichtert das Setzen von chemischen Formeln
\usepackage{fancyvrb} % erlaubt u.a. verbatim in Bilduntersschrift per \SaveVerb{term}|test| und \protect\UseVerb{term}

%-------  Format und Layout
\usepackage{fancyhdr} % ermöglicht den fancy Stil mit Linien für Seitenkopf und -fuß

% Mit dem Paket geometry wird Einfluss auf die Größe der Randbereiche genommen.
% Ich habe als Schwabe in mir die starke Tendenz, die Seiten vollzustopfen und Papier zu sparen.
% Aber dennoch sollte man auch auf Profis höre, die Wahrnehmungsvorgänge beim Lesen erforschen.
% Wenn Zeilen extrem lang werden, dann merkt man ja auch selber, dass man beim Sprung in die
% nächste Zeile stolpern kann. Deshalb ist Nachsicht mit menschlichen Gehirnen angeraten, und
% man denke auch an die Buchbinder, die noch was abschnibbeln müssen.
\usepackage[top=1.5in,bottom=1.5in,inner=1.5in,outer=1.25in,headheight=28pt]{geometry}

%=======  \begin{document}  =====================================================
\begin{document}

%-------  Titelseite  -----------------------------------------------------------
\pagenumbering{nonumbering}
\pagestyle{empty}
\begin{titlepage}
\begin{center}
\vspace*{0.4cm}
\textbf{\LARGE
Eine Vollkorn-Diss: \\
\vspace*{0.2cm} 
In kleinen Git-Schritten \\
von der Hallo-Welt-Saat zu \\
einem LaTeX-Template mit vielen \\
Vitaminen und Ballaststoffen \\
}
\vspace*{2.5cm}
Diplomarbeit in TeX-genese\\
\vspace*{0.3cm}
von\\
\vspace*{0.3cm}
{\large \textbf{Felix Caligula Simplicissimus}}\\
\vspace*{1.5cm}

Referent: Prof. Dr.  A. Schopenhauer\\
Korreferent: Prof. Dr. J. Kepler\\

\vspace*{4.8cm}
angefertigt am\\
Institut für Schriftsatz \\
Universität zu Babel \\
\vspace*{0.8cm}
Juli 2022
\end{center}

\end{titlepage}

\cleardoublepage

%-------  Zusammenfassung  ------------------------------------------------------
\frontmatter

\pagestyle{fancy} % lässt Linien zur Abgrenzung von Kopf- und Fußbereich entstehen
\renewcommand{\headrulewidth}{0.4pt}
\renewcommand{\footrulewidth}{0.4pt}

\chead[]{}
\lhead[\textsc{Zusammenfassung}]{}         
\rhead[]{\textsc{Zusammenfassung}}
\cfoot[]{}
\lfoot[\thepage]{}
\rfoot[]{\thepage}

\chapter{Zusammenfassung}%\label{zusammenfassung}
Diese LaTeX-Vorlage soll einen einfachen Ursprung haben, in der Form eines Hallo-Welt-Dokuments. Ich möchte die Vorlage daraufhin Schritt für Schritt wachsen und komplexer werden lassen. Da ich das ganze as Git-Repositorium ins Netz stelle, kann jede Leserin und jeder Leser die Wachstumsschritte Commit für Commit nachvollziehen. Ich will in den Commitnotizen hilfreiche kurze Beschreibungen dessen geben, was in jedem Beitrag neu hinzu kommt.

\cleardoublepage

%-------  Inhaltsverzeichnis  ----------------------------------------------------
\lhead[\textsc{Inhaltsverzeichnis}]{}       
\rhead[]{\textsc{Inhaltsverzeichnis}}

\tableofcontents
\cleardoublepage

%-------  Hauptteil - die Kapitel  -----------------------------------------------
\mainmatter

\chead[]{} 
\lhead[\textsc{\leftmark}]{}  
\rhead[]{\textsl{\rightmark}}
\cfoot[]{}
\lfoot[\textsc{Seite \thepage}]{\textsc{}}
\rfoot[\textsc{}]{\textsc{Seite \thepage}}

\chapter{Zunächst mal ein Bisschen Text}%\label{nur_text}
Jetzt sollten erstmal ein paar Sätze Text kommen. Wir wollen sehen, wie TeX den Text setzt. Wie werden die Zeilenumbrüche verteilt? Welche Verteilung von Wortabständen entsteht? Beispielsweise müsste man jetzt schon erkennen können, dass in der englischen Version der angelsächsischen Tradition gefolgt wird, dass die Lücke hinter dem Punkt am Satzende etwas größer ist als die übrigen Wortlücken. In der deutssprachigen Tradition gibt es keine spezielle Vergrößerung der Wortlücke hinter dem Satzende.

Als nächstes möchten wir sehen, wie Absätze im Text gehandhabt werden. Deshalb hier also ein durch Leerzeile im Quellcode induzierter neuer Absatz, ein ziemlich kurzer Absatz um genau zu sein.

Mit interessiertem Blick kann man sich natürlich noch weiteren Details zuwenden. Gibt es jetzt schon automatische Wörtertrennung, ohne jegliches Zutun des Autors? Vielleicht und möglicherweise ist das provozierbar durch dedizierte willentliche Verwendung länglichausufernder Vielsilbenzusammensetzungskombinationswörter. Unbedingterweise, unausweichlicherweise, muss jetzt zuerstmal kompiliert werden, um das feststellen zu können. Und das geht ganz einfach per Befehlszeile \glqq pdflatex diss.tex\grqq{} in einer Linuxkonsole.

Was ist eine Ligatur? Das ist wenn Buchstaben miteinander verbunden werden, beispielsweise die jeweils ersten zwei Buchstaben in \glqq fischen\grqq{} \& \glqq fliegen\grqq{}.

Beispielsweise, beispielsweise ... warum nicht eine Abkürzung nehmen, z. B. diese hier? Wenn man sowas bei aktiviertem englischen Schriftsatzmodus macht, dann erkennt man ein erstes Problem: TeX behandelt dies wie ein Satzende und es entstehen viel zu lange Lücken dahinter. Auch kann es äußerst unschön sein, wenn ein Zeilenumbruch die Buchstabenkombination einer Abkürzung auseinanderreißt und die flüssige Lesbarkeit leidet. Wie bekommen wir Abhilfe bei solchen Wehwehchen? Dazu später mehr.

\section{Kapitelabschnitte und Unterabschnitte}

Der obige Titel repräsentiert einen Kapitelabschnitt (section).

\subsection{Kapitelabschnitte und Unterabschnitte}

Der obige Titel repräsentiert einen Unterabschnitt (subsection).

\subsubsection{Kapitelabschnitte und Unterabschnitte}

Der obige Titel repräsentiert einen Unterunterabschnitt (subsubsection).

\section{Aufzählungspunkte und andere Listenarten}

Hier eine erste kleine Liste hilfreicher Resourcen und Werkzeuge:
\begin{itemize}
\item die Webseite ctan.org, dort kann man herausfinden, wofür die verschiedenen Pakete gut sind und was es für Pakete gibt
\item \glqq The Not So Short Introduction To LaTeX 2e\grqq{} von Tobias Oetiker, Hubert Partl, Irene Hyna, Elisabeth Schlegl (einfach nach lshort.pdf suchen)
\item Kile -- ein hübscher LaTeX-Editor für Linux
\item MiKTeX -- ein hübscher LaTeX-Editor inklusive Paketverwaltung für Windows
\end{itemize}

Aufzählung mit Bindestrich:
\begin{itemize}
\item[-] Viren
\item[-] Bakterien
\end{itemize}

gezählt:
\begin{enumerate}
\item Viren als Vektor
\item Tröpfchen aus Lipidmembran als Vektor
\end{enumerate}

kombiniert:
\begin{itemize}
\item Viren
\begin{itemize}
\item[-] Coronaviren
\item[-] Adenoviren
\end{itemize}
\item Bakterien
\begin{itemize}
\item[-] Streptokokken
\item[-] Staphylokokken
\end{itemize}
\end{itemize}

Das Wort selbst als Aufzählungssymbol:
\begin{itemize}
\item[Bakterium] atmet, frißt, lebt
\item[Virus] kein Stoffwechsel, lebt selbst nicht, wird als Bausatz von lebenden Zellen zusammengesetzt
\end{itemize}

\chapter{Mathe im Text}%\label{mathe_im_text}

Die Machzahl $M=v/c$ ist das Verhältnis zwischen lokaler Stömungsgeschwindigkeit und Schallgeschwindigkeit. Oft lässt sie sich als Indikator zur Unterscheidung physikalischer Regime heranziehen. Im Fall der implodierenden Kavitationsblase kann man im Bereich niedriger Machzahlen ($\dot{R}\ll c$) davon ausgehen, dass keine steilen Druck- oder Dichtegradienten innerhalb oder außerhalb der Blase vorkommen. Aber wenn sich die Geschwindigkeit der Blasenwand $\dot{R}$ der Schallgeschwindigkeit nähert oder sie überschreitet (eigentlich sind es innen und außen zwei verschiedene Schallgeschwindigkeiten), dann stellt sich die Situation ganz anders dar, dann entstehen steile Gradienten oder gar Schockwellen.

Die Wahrscheinlichkeit, das beste Viertel $n$ mal hintereinander zu verpassen, ist die Grundlage zur Berechnung der Wahrscheinlichkeit $P$, innerhalb der $n$ Versuche wenigstens einmal das beste Viertel zu erwischen:
\begin{equation}
P=1-\left( \frac{3}{4} \right)^n
\end{equation}

%%% equation* mit Sternchen ohne Nummerierung ist mit den bisher importierten Paketen noch nicht möglich
% Die gleiche Gleichung ohne Nummerierung:
% \begin{equation*}
% P=1-\left( \frac{3}{4} \right)^n
% \end{equation*}

\textbf{Zahlen mit Einheiten:} Zwei Sachen sollten verhindert werden, zum einen der Zeilenumbruch zwischen Zahl und Einheit, zum anderen die Streckung des Abstandes bei locker besetzten Zeilen. Das Paket \textbf{siunitx} kann uns hierbei helfen. Hier eines der Probleme: eine Reisestrecke durch Gebirge von 325 km Länge. Und so sieht die Alternative mit siunitx aus: die Strecke dürfte eine Länge zwischen \num{300} und \SI{350}{\kilo\meter} haben. Wie bei jedem Paket, kann man noch viel mehr Tips und Tricks entdecken, wenn man sich von ctan.org die Doku holt und überfliegt. Als Beispiele \SI[per-mode=symbol]{1.99}[\$]{\per\kilogram} und \SI[per-mode=fraction]{1,345}{\coulomb\per\mole}.

\textbf{Isotope:} Radioaktive Zerfälle können auch in der Form von Gleichungen ausgedrückt werden. Als Beispiel der Zerfall von Tritium:
\begin{equation}
  \isotope{T}  \quad \rightarrow \quad \isotope[3]{He}\,+\,e^{-}\,+\,\bar{\nu}_e  \label{eq_T_decay}
\end{equation}
Hierbei findet das Paket \textbf{isotope} Verwendung, um \isotope[3]{He} zu setzen. Und noch ein Beispiel
\begin{equation}
  \isotope[235][92]{U} + n \quad \rightarrow \quad \isotope[139][56]{Ba} + \isotope[95][36]{Kr} + 2n
\end{equation}
welches neben der Massenzahl auch noch die Kernladungszahl (Protonenzahl) anführt.

\textbf{Chemische Formeln:} Bei der anthropogenen Klimaerwärmung spielen \ce{CO2} und \ce{CH4} zentrale Rollen.




\end{document}
 
