\chapter{Mathe im Text}%\label{mathe_im_text}

Die Machzahl $M=v/c$ ist das Verhältnis zwischen lokaler Stömungsgeschwindigkeit und Schallgeschwindigkeit. Oft lässt sie sich als Indikator zur Unterscheidung physikalischer Regime heranziehen. Im Fall der implodierenden Kavitationsblase kann man im Bereich niedriger Machzahlen ($\dot{R}\ll c$) davon ausgehen, dass keine steilen Druck- oder Dichtegradienten innerhalb oder außerhalb der Blase vorkommen. Aber wenn sich die Geschwindigkeit der Blasenwand $\dot{R}$ der Schallgeschwindigkeit nähert oder sie überschreitet (eigentlich sind es innen und außen zwei verschiedene Schallgeschwindigkeiten), dann stellt sich die Situation ganz anders dar, dann entstehen steile Gradienten oder gar Schockwellen.

Die Wahrscheinlichkeit, das beste Viertel $n$ mal hintereinander zu verpassen, ist die Grundlage zur Berechnung der Wahrscheinlichkeit $P$, innerhalb der $n$ Versuche wenigstens einmal das beste Viertel zu erwischen:
\begin{equation}
P=1-\left( \frac{3}{4} \right)^n
\end{equation}

%%% equation* mit Sternchen ohne Nummerierung ist mit den bisher importierten Paketen noch nicht möglich
% Die gleiche Gleichung ohne Nummerierung:
% \begin{equation*}
% P=1-\left( \frac{3}{4} \right)^n
% \end{equation*}

\textbf{Zahlen mit Einheiten:} Zwei Sachen sollten verhindert werden, zum einen der Zeilenumbruch zwischen Zahl und Einheit, zum anderen die Streckung des Abstandes bei locker besetzten Zeilen. Das Paket \textbf{siunitx} kann uns hierbei helfen. Hier eines der Probleme: eine Reisestrecke durch Gebirge von 325 km Länge. Und so sieht die Alternative mit siunitx aus: die Strecke dürfte eine Länge zwischen \num{300} und \SI{350}{\kilo\meter} haben. Wie bei jedem Paket, kann man noch viel mehr Tips und Tricks entdecken, wenn man sich von ctan.org die Doku holt und überfliegt. Als Beispiele \SI[per-mode=symbol]{1.99}[\$]{\per\kilogram} und \SI[per-mode=fraction]{1,345}{\coulomb\per\mole}.

\textbf{Isotope:} Radioaktive Zerfälle können auch in der Form von Gleichungen ausgedrückt werden. Als Beispiel der Zerfall von Tritium:
\begin{equation}
  \isotope{T}  \quad \rightarrow \quad \isotope[3]{He}\,+\,e^{-}\,+\,\bar{\nu}_e  \label{eq_T_decay}
\end{equation}
Hierbei findet das Paket \textbf{isotope} Verwendung, um \isotope[3]{He} zu setzen. Und noch ein Beispiel
\begin{equation}
  \isotope[235][92]{U} + n \quad \rightarrow \quad \isotope[139][56]{Ba} + \isotope[95][36]{Kr} + 2n
\end{equation}
welches neben der Massenzahl auch noch die Kernladungszahl (Protonenzahl) anführt.

\textbf{Chemische Formeln:} Bei der anthropogenen Klimaerwärmung spielen \ce{CO2} und \ce{CH4} zentrale Rollen.


