\chapter{Mathe im Text}%\label{mathe_im_text}

Die Machzahl $M=v/c$ ist das Verhältnis zwischen lokaler Stömungsgeschwindigkeit und Schallgeschwindigkeit. Oft lässt sie sich als Indikator zur Unterscheidung physikalischer Regime heranziehen. Im Fall der implodierenden Kavitationsblase kann man im Bereich niedriger Machzahlen ($\dot{R}\ll c$) davon ausgehen, dass keine steilen Druck- oder Dichtegradienten innerhalb oder außerhalb der Blase vorkommen. Aber wenn sich die Geschwindigkeit der Blasenwand $\dot{R}$ der Schallgeschwindigkeit nähert oder sie überschreitet (eigentlich sind es innen und außen zwei verschiedene Schallgeschwindigkeiten), dann stellt sich die Situation ganz anders dar, dann entstehen steile Gradienten oder gar Schockwellen.

Die Wahrscheinlichkeit, das beste Viertel $n$ mal hintereinander zu verpassen, ist die Grundlage zur Berechnung der Wahrscheinlichkeit $P$, innerhalb der $n$ Versuche wenigstens einmal das beste Viertel zu erwischen:
\begin{equation}
P=1-\left( \frac{3}{4} \right)^n
\end{equation}

%%% equation* mit Sternchen ohne Nummerierung ist mit den bisher importierten Paketen noch nicht möglich
% Die gleiche Gleichung ohne Nummerierung:
% \begin{equation*}
% P=1-\left( \frac{3}{4} \right)^n
% \end{equation*}
