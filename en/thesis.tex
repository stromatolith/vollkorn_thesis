\documentclass[11pt,a4paper,twoside,titlepage]{book}
%\documentclass[11pt,a4paper,twoside,titlepage,draft]{book}


%-------  maths -- a set of usual suspects
% AMS packages need to be loaded before fontenc, see lshort.pdf v5.01, footnote on page 33
% AMS, that's the American Mathematical Society
\usepackage{amsmath}		% maths stuff, e.g. multi-liner equations aligned or not
\usepackage{amsfonts}		% fonts from the American Mathematical Society
\usepackage{amssymb}		% symbols from the American Mathematical Society
\usepackage{dsfont}			% double-stroke e.g. \mathds{R} for a decent symbol for real numbers
%\usepackage{wasysym}		% double closed integrals with \oiint (but very upright and not looking as nice as AMS integrals)
%\usepackage{esint}			% double closed integrals with \oiint or \varoiint, and of better shape than wasysym

% I could observe that even before ever having explicitly demanded \usepackage{amsmath}
% it was already pulled in by the compiler
 
% context note: I'm compiling on a Linux system (Kubuntu 18.04),
% and at this point simply in a bash console with the command: pdflatex thesis.tex

%-------  input encoding
% I'm working on Linux, maybe slight adjustments have to be made here for Windows
\usepackage[utf8]{inputenc}
\usepackage[OT2, T1]{fontenc}

%-------  font choice
% the package lmodern brings in the Latin Modern family of fonts
% here some related links I noted down a long time ago
% https://tex.stackexchange.com/questions/1390/latin-modern-vs-cm-super
% https://tug.org/FontCatalogue/latinmodernroman/
% https://www.ctan.org/tex-archive/fonts/lm/
\usepackage{lmodern}
 
%-------  document languages
\usepackage[german,main=english]{babel}
\usepackage{lgreek}			% write Greek with \begin{greek} and \end{greek}

%-------  units, chemical formulae, isotopes
\usepackage{siunitx}
\usepackage{isotope}
\usepackage[version=4]{mhchem} % package for chemical equation typesetting
\usepackage{fancyvrb} % enables i.a. verbatim inside figure captions via \SaveVerb{term}|test| and \protect\UseVerb{term}

%-------  format and layout
\usepackage{fancyhdr} % allows fancy page style with lines designating header and footer

%=======  \begin{document}  =====================================================
\begin{document}

%-------  titel page  -----------------------------------------------------------
\pagenumbering{nonumbering}
\pagestyle{empty}
\begin{titlepage}
\begin{center}
\vspace*{0.4cm}
\textbf{\LARGE
A whole-grain thesis: \\
\vspace*{0.2cm}
Letting a Hello-World seed grow \\
inside Git into a LaTeX template rich \\
in minerals and vitamins \\
}
\vspace*{2.5cm}
Master thesis in TeX genesis\\
\vspace*{0.3cm}
by\\
\vspace*{0.3cm}
{\large \textbf{Felix Caligula Doolittle}}\\
\vspace*{1.5cm}

Advisor: Dr. A. Hammurabi\\
Korreferent: Prof. Dr. I. Asimov\\

\vspace*{4.8cm}
completed at the\\
Institute of Cuneiform \\
University of Babylon\\
\vspace*{0.8cm}
July 2022
\end{center}

\end{titlepage}

\cleardoublepage

%-------  abstract  -------------------------------------------------------------
\frontmatter

\pagestyle{fancy} % switch to fancy page style with lines designating header and footer
\renewcommand{\headrulewidth}{0.4pt}
\renewcommand{\footrulewidth}{0.4pt}

\chead[]{}
\lhead[\textsc{Abstract}]{}         
\rhead[]{\textsc{Abstract}}
\cfoot[]{}
\lfoot[\thepage]{}
\rfoot[]{\thepage}

\chapter{Abstract}%\label{abstract}
This LaTeX thesis template begins simple and I am going to let it grow and get more complex step by step. I am sharing this template in the form of a Git repository, thus as a reader you can trace back the growth commit by commit. I want to write useful commit messages, so you can take them each time as a short description of what was added.

\cleardoublepage

%-------  content  --------------------------------------------------------------
\lhead[\textsc{Contents}]{}       
\rhead[]{\textsc{Contents}}

\tableofcontents
\cleardoublepage

%-------  main part - the chapters  ---------------------------------------------
\mainmatter

\chead[]{} 
\lhead[\textsc{\leftmark}]{}  
\rhead[]{\textsl{\rightmark}}
\cfoot[]{}
\lfoot[\textsc{Page \thepage}]{\textsc{}}
\rfoot[\textsc{}]{\textsc{Page \thepage}}

\chapter{Let's begin with text}%\label{just_text}
Now we would of course like to see and read a few lines of text set by TeX. How are the line breaks distributed? What variety of word distances will we get? For instance, it should already now become visible, that in the English version according to Anglo-Saxon tradition a wider word gap is used behind the period at the end of a sentence as compared to the surrounding word spaces. In the German typesetting tradition spaces behind full stops are left the same as the other spaces in the line.

Next, we would like to see how paragraphs are handled by Tex. This here is a new and very short paragraph induced by empty lines in the source code before and after it.

Rubbing our nose further across the text we might discover other interesting details, for example automatic hyphenation which TeX does for you dynamically. Word separation is easier to provoke in the German version where the language allows infinite word combinations. Ligatures are yet another interesting detail, it is the merging of letters, for example the first pairs of letters in the words "fish" \& "fly". Of course, you first have to compile a document from the source code in order to be able to see these things, but it's easy in any Linux console, just type "pdflatex thesis.tex".

For example, for example .... e. g. this abbreviation reveals a first problem: TeX is tricked into adding the larger space behind the period. What can we do about that? It is also not nice if a line break happens to separate abbreviation initials. Readability will surely suffer. Let's get back to these things later.

\section{Sections and subsections}

The above title is a section title.

\subsection{Sections and subsections}

The above title is a subsection title.

\subsubsection{Sections and subsections}

The above title is a sub-sub-section title.

\section{Bullet points and other listings}

Here is a first quick list of useful resources and tools:
\begin{itemize}
\item the website ctan.org, where you can quickly learn what different packages are for
\item "The Not So Short Introduction To LaTeX 2e" by Tobias Oetiker, Hubert Partl, Irene Hyna, Elisabeth Schlegl (search for lshort.pdf)
\item Kile -- a nice LaTeX editor for Linux
\item MiKTeX -- provides a nice editor and package administration for Windows
\end{itemize}


with dashes instead of bullet points:
\begin{itemize}
\item[-] virus
\item[-] bacterium
\end{itemize}

counted:
\begin{enumerate}
\item virus as vector
\item lipid membrane droplet as vector
\end{enumerate}

nested:
\begin{itemize}
\item viruses
\begin{itemize}
\item[-] Coronavirus
\item[-] Adenovirus
\end{itemize}
\item bacteria
\begin{itemize}
\item[-] Streptococcus
\item[-] Staphylococcus
\end{itemize}
\end{itemize}

the word itself used as item symbol:
\begin{itemize}
\item[bacterium] breathes, feeds, lives
\item[virus] no metabolism, does not live, created as assembled kit by living cells
\end{itemize}


\chapter{Math in the text}%\label{math_in_text}

The Mach number $M=v/c$ is the ratio between the local fluid velocity and the speed of sound and can often be taken as an indicator for the transition between different physical regimes. In the case of imploding cavitation bubbles the low-Mach regime ($\dot{R}\ll c$) means that there are no sharp pressure gradients inside or outside the bubble. But when the interface speed $\dot{R}$ approaches, or exceeds, the speed of sound in the gas or liquid phase the situation changes and gradients or even shock waves will arise.

The chances of missing the best quarter of cases $n$ times in a row is the basis when calculating the probability $P$ of getting something from within the best quarter once among the $n$ trials:
\begin{equation}
P=1-\left( \frac{3}{4} \right)^n
\end{equation}

%%% equation* without numbering is not yet possible with the current set of imported packages
% The same equation without numbering:
% \begin{equation*}
% P=1-\left( \frac{3}{4} \right)^n
% \end{equation*}

\textbf{Numbers with units:} We would like to see to things avoided, on the one hand line breaks between number and unit, on the other hand a stretching of the space in between in sparsely filled lines. The package \textbf{siunitx} can help us. Here one of the problems: a long travel distance of 325 km length. And this is how the alternative with siunitx works: The covered distance should be between \num{300} and \SI{350}{\kilo\meter} at the end. As with any other package, you can discover many more tips and tricks after fetching the docs from ctan.org and scrolling quickly through the pages. Als examples \SI[per-mode=symbol]{1.99}[\$]{\per\kilogram} and \SI[per-mode=fraction]{1,345}{\coulomb\per\mole}.

\textbf{Isotopes:} Radioactive decays can also be expressed in the form of equations. Here the decay of tritium as example
\begin{equation}
  \isotope{T}  \quad \rightarrow \quad \isotope[3]{He}\,+\,e^{-}\,+\,\bar{\nu}_e,  \label{eq_T_decay}
\end{equation}
whereby, the package \textbf{isotope} is used for typesetting \isotope[3]{He}. As another example one possible variant of the neutron-induced fission of uranium-235
\begin{equation}
  \isotope[235][92]{U} + n \quad \rightarrow \quad \isotope[139][56]{Ba} + \isotope[95][36]{Kr} + 2n
\end{equation}
depicting besides the mass number also the atomic number, i.e. the number of protons in the core.

\textbf{Chemical formulae:} Two most important drivers of anthropogenic climate warming are \ce{CO2} and \ce{CH4}.
% let's use fancyvrb to explain the \ce command provided by mhchem in the text
The Package \textbf{mhchem} provides the short command \Verb+\ce+ allowing to write simple (\Verb+\ce{H2O}+ turns into \ce{H2O}) or more complicated (\Verb|\ce{[\{(X2)3\}2]^3+}| turns into \ce{[\{(X2)3\}2]^3+}) formulae. Grabbing the example of a reaction equation from the package docs, \Verb?\ce{Hg^2+ ->[I-] HgI2 ->[I-] [Hg^{II}I4]^2-}?, we can compile again and convince ourselves that it works out just fine:
\begin{equation}
\ce{Hg^2+ ->[I-] HgI2 ->[I-] [Hg^{II}I4]^2-}
\end{equation}







\end{document}

