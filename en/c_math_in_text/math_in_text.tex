\chapter{Math in the text}%\label{math_in_text}

The Mach number $M=v/c$ is the ratio between the local fluid velocity and the speed of sound and can often be taken as an indicator for the transition between different physical regimes. In the case of imploding cavitation bubbles the low-Mach regime ($\dot{R}\ll c$) means that there are no sharp pressure gradients inside or outside the bubble. But when the interface speed $\dot{R}$ approaches, or exceeds, the speed of sound in the gas or liquid phase the situation changes and gradients or even shock waves will arise.

The chances of missing the best quarter of cases $n$ times in a row is the basis when calculating the probability $P$ of getting something from within the best quarter once among the $n$ trials:
\begin{equation}
P=1-\left( \frac{3}{4} \right)^n
\end{equation}

%%% equation* without numbering is not yet possible with the current set of imported packages
% The same equation without numbering:
% \begin{equation*}
% P=1-\left( \frac{3}{4} \right)^n
% \end{equation*}


